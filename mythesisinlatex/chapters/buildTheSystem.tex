\chapter{Build the System}\label{chapter:build}

In the next chapter, a detail description of the process of the system development following the methodology introduced before and how the methodology was adapted and implemented are presented. 

\section{Acquiring Context Relevance} \label{sec:acr}

The first step of the methodology is to discover the relevance of the contextual factors to the current implementation domain (i.e. clothes shopping). 

To adapt the recommendations to the user's current contextual situation requires an understanding of the relationship between user preferences and contextual conditions. Thus it is proposed that explicit user ratings or any form of preferences should be given under several different contextual conditions. For instance, the user must rate a given clothes item when the temperature is hot, warm and cold. It is quite time and resource consuming because the user needs to only give ratings after they have experienced the context. Therefore, to reduce the risk of collecting data for unimportant context factors, an experiment needs to be first set up to determine which context factors are interesting to study \cite{ref:18}. 

The experiment should be able to investigate how the influence of each contextual factors changes on user's purchasing decision for different clothes types and to provide quantitative measurements so that they can be used as weight in the similarity measurement in the following retrieving algorithm. Considering all above reasons, the methodology developed in paper \cite{ref:18} was adopted to assess the context relevance.

This methodology is based on a web tool for acquiring context relevance judgements and a statistical data analysis method to quantitatively measure the influence of each contextual conditions on different clothes types. Following this methodology, a web survey was designed and developed in this thesis. First, an initial set of contextual factors and conditions (values for the factors) were selected referring to some existing literatures about context-aware applications \cite{ref:5, ref:12, ref:19}. The selected contextual factors and conditions are listed in Table \ref{tab:factors}. Then, the clothes items were retrieved from Zalando \footnote{http://www.zalando.co.uk/}. Especially, clothes of these brands are collected: Marc O'Polo, Tom Tailor, Esprit, S.Oliver, Benetton. Because these five brands are in the middle price category and are well known and generally acceptable by most people. Moreover, the types and number of clothes offered by these brands are similar to each other and are rich enough to cover most common clothes types. After the clothes were retrieved, they were aggregated into a relatively small list of categories so that the problem of data sparseness can be avoided. Totally, 14 categories were defined: tops, dresses, underwear, cardigans, trousers, coats, blouses, jackets, skirts, jeans, socks, swimwear, suits and shirts.

\begin{table}[H]
	\centering
	\caption{Context factors used in the Web survey}
	\label{tab:factors}
	\begin{tabular}{p{1.1in}p{1.6in}|p{1.1in}p{1.6in}} \hline
		Context Factor & Conditions  & Context Factor & Conditions  \\ \hline 
		budget & budget buyer & purchasing intent & work \\
 		& high spender &  & daily wear \\
 		& price-for-quality buyer &  & party \\
		time of the day & morning time &  & sports \\
 		& afternoon &  & no special \\
 		& night time & companion & with girl-/boy-friend \\
		day of the week & weekend &  & with family  \\
		& working day &  & with children \\
		crowdedness & shop is crowded &  & alone \\
		& shop is not crowded &  & with friends \\
		& shop is empty & weather  & snowing \\
		mood & shopaholic &  & clear sky \\
		& outdoorsy &  & sunny \\
		& like a party animal &  & raining \\
		& normal &  & cloudy \\	
		season & spring & transport & walking \\
 		& summer &  & public transport \\
 		& autumn &  & bicycle \\
 		& winter &  & car \\
		temperature & warm & time available & half day \\
 		& cold &  & one day \\
 		& hot \\
	\end{tabular}
\end{table}


After the data was prepared, a simple web application was developed for acquiring the relevance of the selected contextual factors for the clothes categories (see Figure 1). In the web application, the user will be randomly given a clothes category and will be asked to imagine themselves being under a randomly chosen contextual condition and then choose the influence of the selected contextual condition on their intention to buy the selected type of clothes. As an example of the questions posed to the user consider the situation depicted in Figure 1. Here we first ask the user to imagine a typical shopping scenario: ?Imagine that you are in Munich and you are doing offline shopping for clothes. You are thinking about buying Skirts.? Then the user is asked to select the influence (i.e., positive, no effect, negative) of the three randomly chosen contextual conditions on their decision to buy the clothes. As an example of a contextual condition: "Imagine that the weather is cloudy." Every user was requested to interact with at least 10 of these pages (as in Figure 1). 

38 participants took part in this web survey. Overall 1190 responses were given to one of the questions shown in Figure 1. Because no pre-knowledge is known whether certain context conditions are more likely to influence user?s decision, we sampled the value of clothes categories and contextual conditions using uniform distribution so that all possible values can be reached with equal opportunities.

\subsection{Analysis of Context Relevance} \label{sec:acr_acr}

The goal of this web survey in this thesis is to find out quantitatively how the context factors influence user decisions whether or not to buy clothes from different categories. 

The web survey delivered samples for the distribution $P(I|C_i,T)$ where $I$(Influence) is the response variable taking one of the three values: positive, negative, or no effect, $T$ is a clothes category (e.g., tops, skirts), and $C_1, ... C_N$ are the context factors that may or may not influence the user decision. This distribution models the influence of the context factors on the user?s decision considering different clothes categories. 

The spread of a categorical variable $X={x_1, ... x_n}$ can be measured by looking at the entropy of the random variable. If $P(X=x_i)=\pi_i$, the entropy of $X$ is: 
$$E(X)=-\sum_{1\leq i\leq n}\pi_i \cdot log{\pi_i} $$

This measurement of the spread can be used to estimate the association between variable $X_1$:  user?s intention to buy a certain item (i.e., positive, negative or no effect) and variable $X_2$: one of the current context factor (e.g., current budget). Informally, if the influence of the context factor is strong, then the spread of variable $X_1$will be reduced if $X_2$ is known, and it is weak if the spread of $X_1$ remains unchanged even if $X_2$ is known and this association can be formally defined as [18]:
$$U=\frac{E(X_1)-\varepsilon(E(X_1|X_2))}{E(X_1)}$$
where $E(X_1)-\varepsilon(E(X_1|X_2))$ is the difference between the spread of $X_1$ and the expected spread of $X_2$ which measures the influence of the context factor to user?s decision, where $\varepsilon(X)$ denotes the expected value of the random variable $X$. As the spread of $(X_1|X_2)$ is the same as the spread of $X_1$. We computed $U$ for all context factors and clothes categories and the ordered factors in descending order of $U$ for each clothes category can be seen in Appendix of this paper.

\textbf TODO Some simple analysis of the result

\section{The Proposed Approach} \label{sec:pa}

In this section, a recommendation approach is built, the goal of which is to integrate the contextual information into the recommendation process and recommend items that might be of interest to user under a specific context situation.

To integrate the contextual information into the recommender system, we are going to adopt the context-driven querying and search approach. As was introduced in section xx.xx, this approach uses contextual information and/or user?s specified interest to query or search a repository of resources and present the most appropriate ones to the user. The repository usually contains resources that are tagged with contextual information while collecting them. Corresponding to this approach, we use case-based recommendation technique to realize it. As was discussed in section xx.xx, case-based recommendation technique is a branch of knowledge-based recommendation. Compared to collaborative filtering and content-based technique, case-based recommendation has no cold-start problem because it can rely on the case base (or knowledge base) for initial recommendation. This case base can be set up using expert experience quickly and does not require pre-training like what is done in paper [5] for model-based approach. 

Each case in the case base is composed of an item and the contextual situation under which the item is bought. Here a contextual situation is a combination of several context factors and their corresponding values. For example, the user may ask for recommendation of clothes for work (condition one) when the weather is warm (condition two). For the recommendation, the user first submit the contextual information as the query. Then the system will search in the case base and select the cases with the most similar context situation and then recommend the items or items similar to the items contained in the cases to the user. 

On the other hand, knowledge-based recommender system has the disadvantages of static suggestion ability because the knowledge-base is usually preset by domain expert and barely changes. Also usually the domain expert and the knowledge engineer are not the same person, the communication cost requires an efficient way for knowledge engineering. Based on those considerations, we extend the case-based recommendation by integrating collaborative filtering approach for case base (knowledge base) setup so that all the application users will play the expert role and their purchased items together with the contextual information will be added to the case base as a new case for future recommendation. Although the collaborative filtering approach is used, the correlation between users is performed at the session level (i.e., each submitted case is independent of itself and will not be related to the user who submit the case). Thus no user identification is required and a considerable amount of example data is not needed for each single user in order to deliver effective recommendations.[26])

\section{Towards Case-Based Recommender Systems} \label{sec:tcbrs}

As was introduced in section xx.xx, Case-Based Recommender Systems (CBRSs) apply case-based reasoning (CBR) methodology to solve recommendation problem by re-using or adapting past recommendation solution stored in past similar cases. In the CBRSs framework introduced in paper [22], a typical recommendation process is composed of six steps: retrieval, reuse, revise, review, retain and iterate. We are going to apply this framework to the building of our context-aware CBRS.

Input: To get a list of recommendation, the user will first submit a query of current contextual information. A query is composed of a logical query with fixed context constraints and a feature value vector of context factors and their corresponding value that the user wishes to be considered in recommendation. For example, if a user is a budget buyer and want to buy clothes for sports purpose when the temperature is hot and the user wants to find clothing items sold in stores that will be still open in the next 30 minutes within 2000 meters, the query will be structured as follows:
\begin{equation} \label{eq:query}
\begin{split}
	query =  \{((distance \leq 2000m)\wedge(timeopen = now+30min)), &\\
	              (budget (budget buyer), intent (sports), temperature (hot))\} &
\end{split}
\end{equation}

Retrieval: After the user submit a context query, contextual factors such as budget, intent etc. will be used to find and rank cases with similar context. The definition of similarity and the similarity assessment algorithm will be introduced in section xx.xx. 

Reuse: In the final recommendation, nine items in the cases will be recommended to the user. However those items will not be ranked only according to the similarity of the cases to the current context. As for initial recommendation of conversational recommendation system in an exploratory mobile scenario, diversity is an important consideration to ensure the coverage of the current scope of candidate items. Thus we extend the bounded greedy selection algorithm to select the cases with the most diverse set of items among the retrieved most similar cases. 

Revise: Before the items are recommended to the user, logical constraints such as distance (e.g. find clothes within 2000 meters) or open time (e.g., shops still open in the next 30 minutes) will be used to check the availability of those items. If for example a recommended item is too far away , then other similar items will be recommended instead.

Review \& Iterate: After the initial recommendations are presented, users can update the recommendations iteratively through critiquing directly on item features. This is also called conversation-based Active Learning strategy and has been explained in detail in section xx.xx.

Retain: Finally, when the user selects and purchases an item, the time together with the current context situation will be stored as a new case in the case base.

\section{Case Model} \label{sec:cm}

The Case Base is made of made of two components: item bought $I$ and context situation $C$: 
\begin{equation} \label{eq:caseMode}
CB = I \times C
\end{equation}

Each case $c=(i,e) \in CB$ in the case base will be consist of two sub-elements $i,e$ which are instances of the spaces $I,C$ respectively. As was introduced before, the cases will not be correlated with the user who submits it, thus user model is not contained in the case in our system. A case is built during a human/machine interaction [26]. In our system, a case is created when the user purchases the item. According to Peak-end Rule introduced in section xx.xx, how a user feels about an experience is highly influenced by the end of the experience. We assume here that users give high rates for the items they buy.  Since the case is created when the user bought an item, we here get rid of the evaluation model as well in the case base. In the following we will introduce these two components in detail. 

$C$ is the data structure that defines the context situation under which the item is bought. It is composed of a feature value vector of context factors and their corresponding value that the user wishes to be considered in recommendation and a feature value vector of context factors and their corresponding factor importance weight. The factor importance weight reflect the level of influence of the context factors for the recommendation of the clothing item contained in the same case. They are determined by the clothing type of the clothing item and has been calculated using experiment in Section xx.xx. For a full list of the factor importance weights for different clothing type please refer to Table xx.xx. The main context factors are: distance, day of the week, temperature, time available, transport, weather, time of the day, crowdedness, intent of purchasing, companion, season and budget. For a detail list of the context factors and their values, please refer to Table XX. For a typical example, if a user is a budget buyer and is looking for clothes for sports when the temperature is hot, the context situation can be structured as follows:
\begin{equation} \label{eq:context}
\begin{split}
	{context}_{attributes} = \{(budget (budget buyer), intent (sports), temperature (hot)), &\\
	                                            ((w_{budget}(0.7), w_{intent}(0.6), w_{temperature}(0.9))\} &
\end{split}
\end{equation}

$I$ is the data structure that describes the clothing item bought by the user. It is represented as a feature value weight vector (Equation xx.xx). It is directly borrowed from the base system introduced in section xx.xx. 

Through this case model, knowledge about what kind of items users buy in a certain context situation can be obtained. To provide recommendation, cases with context situation similar to the current user can be retrieved and the items contained in those cases can be used directly for recommendation. They can also be used as reference items to find other similar items to recommend. Thus we are going to show how the similarity between current context and retrieved cases and similarity between items are calculated.

\section{Similarity Assessment} \label{sec:sa}

To get the similarity between current context and retrieved cases, we borrow the Euclidean Overlap Metric (HEOM) [26, 34]:
\begin{equation} \label{eq:heom}
heom(x,y)=\frac{1}{\sqrt{\sum^n_{i=1}w_i}}\sqrt{\sum^n_{i=1}w_id_i(x_i,y_i)^2}
\end{equation}
where:

\textbf{TODO the equation}

Here $range_i$ is the difference between the maximum and minimum value of a numeric feature, and $overlap(x_i, y_i)=1$ if $x_i \neq y_i$ and 0 otherwise. This metric measures the distance between two vectors. Thus the further away two vectors, the more similar they are. We modified this metric so that it can be applied to our system.

By using the previously discussed case model and query structure, the feature value vectors of context factors describing the context situation in both structures can be fed into the similarity metric. First, for all the context factors submitted in the user? query (currentContext), we will calculate the similarity to the target context factors (targetContext) in the cases. In some cases, the targetContext may not contain context factors that are specified by the user (e.g., the user enables the context factor shopping intent and budget, but the targetContext only contains budget). In some other cases, the currentContext may not contain some context factors contained in the targetContext. So when computing similarities, we use user specified context factors as base and only consider the similarities between context factors specified by the user. The context factors contained in the targetContext but not in the currentContext will be ignored, because the user chooses to ignore those factors. If the targetContext does not contain the context factors specified in currentContext, the similarity will be added by $1* w_i$. ($w_i$ here corresponds to the feature factor weight).

The simplified similarity metric is displayed as follows:
\begin{algorithm}
\caption{The simplified similarity metric}
\label{list:similarity}
\begin{algorithmic}
\Function{getSimilarity}{$query, case$} 
	\State $targetContext \gets getCaseContext(case)$
	\State $currentContext \gets getQueryContext(query)$
	\ForAll{$context factors defined in targetConetxt$}
		\If{the $targetContext$ contains the current context factor in $currentContext$}
			\State $sim \gets sim +  factorSimilarity(\pi_f(targetContext), \pi_f(currentContext))$
			\State $weight \gets weight + getWeight(case,\pi_f(targetContext))$
		\Else
			\State $sim \gets sim +  1*getWeight(case,\pi_f(targetContext))$
			\State $weight \gets weight + getWeight(case,\pi_f(targetContext))$
		\EndIf
	\EndFor
\EndFunction
\end{algorithmic}
\end{algorithm}

After the similarities of the cases retrieved are calculated, they will be first ranked according to the calculated similarity, the most similar the first. Then we use the bounded greedy selection algorithm to select and rank the cases based on the diversity of items contained in those cases. 

\section{Explanation Generation} \label{sec:eg}

The learned factor importance weights contained in each case will be used for generating explanations for the recommendations.  Analyzing the learned importance weights one can generate explanation based on the values of these parameters. More specifically, given a case that includes an item $i$ and a context situation $c$ in which a set of context factors $(c_1, ..., c_k)$ as well as the corresponding factor weights $(w_1, ..., w_k)$ are specified, we first find out the set of context factors $(c\prime_1, ..., c\prime_j)$ that overlap with the context factors specified in user?s query as well as the corresponding factors weights $(w\prime_1, ..., w\prime_j)$. Then we identify a fixed number of the context factors $f, g, h$ (in this thesis, at most three context factors will be identified), among those overlapped context factors, with the highest importance weights $w\prime_f, w\prime_g, w\prime_h$. If the number of overlapped context factors is smaller than the fixed number, then the whole set of overlapped context factors will be used. After those identified context factors is ranked in descending order based on their corresponding weights, they will be used to generate a positive explanation for recommending item $i$. For example, if item $i$ (e.g., a dress) is recommended in the contextual situation ?the shopping purpose is for party, the user is a budget buyer and the temperature is hot? and the overlapped contextual conditions are ?the shopping purpose is for party? and ?the user is a budget buyer?, we observe factor ?budget? has a higher importance than ?shopping purpose?, so we explain that the dress is recommended because ?other user bought similar clothes when the she/he is a budget buyer and the shopping purpose is for party?.

\section{Interaction and Interface Design} \label{sec:eg}

In this section we illustrate the main features of Shopper, a mobile context-aware recommender system. Shopper is a Android application built based on a personalized recommender system using conversion-based active learning strategy developed in paper [30]. Shopper integrates contextual information using case-based recommendation approach into this system so that users can obtain recommendations adapted to the recommendation context. The detail of the case-based recommendation approach has been discussed before in section xx.xx. To get an initial set of recommendations, the user makes a recommendation request specifying contextual conditions and then a list of clothing items (including pictures and descriptions) will be returned. Those recommendations are obtained through searching for items bought by other users under similar items. Then the user can critique and update the recommendations iteratively to get the most satisfying item. We will not describe this system illustrating a typical interaction.

In the initial step of the interaction with Shopper the user normally sets the current shopping context. Figure xx shows the GUI for enabling and setting the values of the selected contextual factors. It is built using Android's Preference APIs so that the settings interface will be consistent with the user experience in other Android apps (including the system setting). Here we can see the user can switch on/off some of these factors, e.g., ?Time of the day?  or ?Weather? using checkbox. When these factors are switched on the recommender system will take into account their current values (conditions) by querying a third party service. For example, to get the current weather and temperature, the system will query the Yahoo weather API and parse the returned XML file for the target information. For some other context factors, e.g., ?Budget? or ?Intent?, the user can switch them on or off by selecting the ?Off? option in a pop up dialog. When those factors are switched on, the user need to provide the value manually by selecting among the options in a pop up dialog as in Figure xx (right). After the user close the dialog, the selected value will be shown as subtitle under the corresponding item line so that the user can easily have a clear view of what are enabled and selected. The full set of contextual factors is the same as in the web application described earlier and their values could be found in Table xx. The contextual conditions: distance, day of the week, temperature, weather, time of the day, crowdedness and season are automatically obtained from third party services. The remaining contextual conditions, if the user has enabled them, must be entered manually by the user.

After the user has enabled some contextual factors and provided the values for them, the system can be requested for recommendations. A short number of recommendations (nine in this system) will be represented to the user as depicted in Figure xx. If the user is interested in any of the item, she/he can click on the picture and see the detail page of the item as depicted in Figure xx. In the detail view, the user can see an explanation of the reason why this item is recommended to the user because we believe explanation will boost the transparency and user?s trust to the system. A typical explanation can be ?Other user bought similar clothes when they are feeling like a party animal, they are a budget buyer, it is weekend.? Those are the contextual conditions that are most influential for recommending the current clothes item (as was explained in previous section). 

The initial consideration is to put the explanation in the main view so that they can be more directly observed by the user. However, it can be seen that there is already critiques explanation in the main view. Considering the limited space of mobile devices, the user experience will be reduced if too much text is displayed together. Moreover, it might be the case that each item has a different reason for being recommended. For example, if a user requires for recommendation of clothing item for context situation ?budget buyer, for work and for cold temperature?. One case satisfying contextual condition ?budget buyer? and another case satisfying contextual condition ?for work and for cold temperature? can all be recommended to the user in this case. Thus it is better the put the explanation in the detail page of each item separately.

In the detail view, the user will also see an explanation of the location of the clothes as well as a map view of the location of the clothes relative to user?s current location as depicted in Figure xx. For users who only want to find clothes nearby, it is an important factor for purchasing decision. If the user is not satisfied with the recommended clothes, she/he can critique on the item features to iteratively update the recommendations. This step corresponds to the review and iterate step in case-based recommendation as was discussed in section xx.xx. It relies on the main functions of the system in paper [30].

If the user finds the ideal item, she/he can enter the detail view and click on the button ?select and finish? (Figure xx.xx). Then the selected item together with the current contextual situation will be recorded as a new case in the system case base. 




















