\chapter{Build the System}\label{chapter:build}

\section{Acquiring Context Relevance} \label{sec:acr}

The first step of the methodology is to discover the relevance of the contextual factor to our current implementation domain (i.e. clothes shopping). 

To adapt the recommendations to the user?s current contextual situation requires an understanding of the relationship between user preferences and contextual conditions. Thus it is proposed that explicit user ratings or any form of preferences should be given under several different contextual conditions. For instance, the user must rate a given clothing item when the temperature is hot, warm and cold which is quite time and resource consuming because user needs to only give ratings after they have experienced the context. Therefore, to reduce the risk of collecting data for unimportant context factors, an experiment needs to be first set up to determine which context factors are interesting to study. 

We would like to know how the influence of each contextual factors change for different clothes types on user?s purchasing decisions and we also want to have a quantitative measure so that it can be used as weight in the similarity measurement in the following retrieving algorithm. 

Consider all above reasons, we adopt the methodology developed in paper [18] to assess the context relevance.

This methodology is based on a web tool for acquiring context relevance judgements and a statistical data analysis method to quantitatively measure the influence of each contextual conditions on different clothes category. Following this methodology, we designed and developed a web survey. First, an initial set of contextual factors and conditions (values for the factors) were selected referring to some existing literatures about context-aware applications [5, 12, 19]. The selected contextual factors and conditions are listed in Table 1. Then, the clothes items were retrieved from zalando.co.uk. Especially, clothes of these brands are collected: Marc O'Polo, Tom Tailor, Esprit, S.Oliver, Benetton. Because these five brands are in the middle price category and are well known and generally acceptable by most people. Moreover, the types and number of clothes offered by these brands are similar to each other and are rich enough to cover most common clothes types. After the clothes were retrieved, they were aggregated into a relatively small list of categories so that the problem of data sparseness can be avoided. Totally, 14 categories were defined: tops, dresses, underwear, cardigans, trousers, coats, blouses, jackets, skirts, jeans, socks, swimwear, suits and shirts.

After the data was prepared, a simple web application was developed for acquiring the relevance of the selected contextual factors for the clothes categories (see Figure 1). In the web application, the user will be randomly given a clothes category and will be asked to imagine themselves being under a randomly chosen contextual condition and then choose the influence of the selected contextual condition on their intention to buy the selected type of clothes. As an example of the questions posed to the user consider the situation depicted in Figure 1. Here we first ask the user to imagine a typical shopping scenario: ?Imagine that you are in Munich and you are doing offline shopping for clothes. You are thinking about buying Skirts.? Then the user is asked to select the influence (i.e., positive, no effect, negative) of the three randomly chosen contextual conditions on their decision to buy the clothes. As an example of a contextual condition: ?Imagine that the weather is cloudy.? Every user was requested to interact with at least 10 of these pages (as in Figure 1). 

38 participants took part in this web survey. Overall 1190 responses were given to one of the questions shown in Figure 1. Because no pre-knowledge is known whether certain context conditions are more likely to influence user?s decision, we sampled the value of clothes categories and contextual conditions using uniform distribution so that all possible values can be reached with equal opportunities.

\subsection{Analysis of Context Relevance} \label{sec:acr_acr}

The goal of this web survey in this thesis is to find out quantitatively how the context factors influence user decisions whether or not to buy clothes from different categories. 

The web survey delivered samples for the distribution $P(I|C_i,T)$ where $I$(Influence) is the response variable taking one of the three values: positive, negative, or no effect, $T$ is a clothes category (e.g., tops, skirts), and $C_1, ... C_N$ are the context factors that may or may not influence the user decision. This distribution models the influence of the context factors on the user?s decision considering different clothes categories. 

The spread of a categorical variable $X={x_1, ... x_n}$ can be measured by looking at the entropy of the random variable. If $P(X=x_i)=\pi_i$, the entropy of $X$ is: 
$$E(X)=-\sum_{1\leq i\leq n}\pi_i \cdot log{\pi_i} $$

This measurement of the spread can be used to estimate the association between variable $X_1$:  user?s intention to buy a certain item (i.e., positive, negative or no effect) and variable $X_2$: one of the current context factor (e.g., current budget). Informally, if the influence of the context factor is strong, then the spread of variable $X_1$will be reduced if $X_2$ is known, and it is weak if the spread of $X_1$ remains unchanged even if $X_2$ is known and this association can be formally defined as:
$$U=\frac{E(X_1)-\varepsilon(E(X_1|X_2))}{E(X_1)}$$
where $E(X_1)-\varepsilon(E(X_1|X_2))$ is the difference between the spread of $X_1$ and the expected spread of $X_2$ which measures the influence of the context factor to user?s decision, where $\varepsilon(X)$ denotes the expected value of the random variable $X$. As the spread of $(X_1|X_2)$ is the same as the spread of $X_1$. We computed $U$ for all context factors and clothes categories and the ordered factors in descending order of $U$ for each clothes category can be seen in Appendix of this paper.

\textbf TODO Some simple analysis of the result

\section{The Proposed Approach} \label{sec:pa}

In this section, a recommendation approach is built, the goal of which is to integrate the contextual information into the recommendation process and recommend items that might be of interest to user under a specific context situation.

To integrate the contextual information into the recommender system, we are going to adopt the context-driven querying and search approach. As was introduced in section xx.xx, this approach uses contextual information and/or user?s specified interest to query or search a repository of resources and present the most appropriate ones to the user. The repository usually contains resources that are tagged with contextual information while collecting them. Corresponding to this approach, we use case-based recommendation technique to realize it. As was discussed in section xx.xx, case-based recommendation technique is a branch of knowledge-based recommendation. Compared to collaborative filtering and content-based technique, case-based recommendation has no cold-start problem because it can rely on the case base (or knowledge base) for initial recommendation. This case base can be set up using expert experience quickly and does not require pre-training like what is done in paper [5] for model-based approach. 

Each case in the case base is composed of an item and the contextual situation under which the item is bought. Here a contextual situation is a combination of several context factors and their corresponding values. For example, the user may ask for recommendation of clothes for work (condition one) when the weather is warm (condition two). For the recommendation, the user first submit the contextual information as the query. Then the system will search in the case base and select the cases with the most similar context situation and then recommend the items or items similar to the items contained in the cases to the user. 

On the other hand, knowledge-based recommender system has the disadvantages of static suggestion ability because the knowledge-base is usually preset by domain expert and barely changes. Also usually the domain expert and the knowledge engineer are not the same person, the communication cost requires an efficient way for knowledge engineering. Based on those considerations, we extend the case-based recommendation by integrating collaborative filtering approach for case base (knowledge base) setup so that all the application users will play the expert role and their purchased items together with the contextual information will be added to the case base as a new case for future recommendation. Although the collaborative filtering approach is used, the correlation between users is performed at the session level (i.e., each submitted case is independent of itself and will not be related to the user who submit the case). Thus no user identification is required and a considerable amount of example data is not needed for each single user in order to deliver effective recommendations.[26])

\section{Towards Case-Based Recommender Systems} \label{sec:tcbrs}

As was introduced in section xx.xx, Case-Based Recommender Systems (CBRSs) apply case-based reasoning (CBR) methodology to solve recommendation problem by re-using or adapting past recommendation solution stored in past similar cases. In the CBRSs framework introduced in paper [22], a typical recommendation process is composed of six steps: retrieval, reuse, revise, review, retain and iterate. We are going to apply this framework to the building of our context-aware CBRS.

Input: To get a list of recommendation, the user will first submit a query of current contextual information. A query is composed of a logical query with fixed context constraints and a feature value vector of context factors and their corresponding value that the user wishes to be considered in recommendation. For example, if a user is a budget buyer and want to buy clothes for sports purpose when the temperature is hot and the user wants to find clothing items sold in stores that will be still open in the next 30 minutes within 2000 meters, the query will be structured as follows:
\begin{equation} \label{eq:query}
\begin{split}
	query =  \{((distance \leq 2000m)\wedge(timeopen = now+30min)), &\\
	              (budget (budget buyer), intent (sports), temperature (hot))\} &
\end{split}
\end{equation}

Retrieval: After the user submit a context query, contextual factors such as budget, intent etc. will be used to find and rank cases with similar context. The definition of similarity and the similarity assessment algorithm will be introduced in section xx.xx. 

Reuse: In the final recommendation, nine items in the cases will be recommended to the user. However those items will not be ranked only according to the similarity of the cases to the current context. As for initial recommendation of conversational recommendation system in an exploratory mobile scenario, diversity is an important consideration to ensure the coverage of the current scope of candidate items. Thus we extend the bounded greedy selection algorithm to select the cases with the most diverse set of items among the retrieved most similar cases. 

Revise: Before the items are recommended to the user, logical constraints such as distance (e.g. find clothes within 2000 meters) or open time (e.g., shops still open in the next 30 minutes) will be used to check the availability of those items. If for example a recommended item is too far away , then other similar items will be recommended instead.

Review \& Iterate: After the initial recommendations are presented, users can update the recommendations iteratively through critiquing directly on item features. This is also called conversation-based Active Learning strategy and has been explained in detail in section xx.xx.

Retain: Finally, when the user selects and purchases an item, the time together with the current context situation will be stored as a new case in the case base.

\section{Case Model} \label{sec:cm}

The Case Base is made of made of two components: item bought $I$ and context situation $C$: 
\begin{equation} \label{eq:caseMode}
CB = I \times C
\end{equation}

Each case $c=(i,e) \in CB$ in the case base will be consist of two sub-elements $i,e$ which are instances of the spaces $I,C$ respectively. As was introduced before, the cases will not be correlated with the user who submits it, thus user model is not contained in the case in our system. A case is built during a human/machine interaction [26]. In our system, a case is created when the user purchases the item. According to Peak-end Rule introduced in section xx.xx, how a user feels about an experience is highly influenced by the end of the experience. We assume here that users give high rates for the items they buy.  Since the case is created when the user bought an item, we here get rid of the evaluation model as well in the case base. In the following we will introduce these two components in detail. 

$C$ is the data structure that defines the context situation under which the item is bought. It is composed of a feature value vector of context factors and their corresponding value that the user wishes to be considered in recommendation and a feature value vector of context factors and their corresponding factor importance weight. The factor importance weight reflect the level of influence of the context factors for the recommendation of the clothing item contained in the same case. They are determined by the clothing type of the clothing item and has been calculated using experiment in Section xx.xx. For a full list of the factor importance weights for different clothing type please refer to Table xx.xx. The main context factors are: distance, day of the week, temperature, time available, transport, weather, time of the day, crowdedness, intent of purchasing, companion, season and budget. For a detail list of the context factors and their values, please refer to Table XX. For a typical example, if a user is a budget buyer and is looking for clothes for sports when the temperature is hot, the context situation can be structured as follows:
\begin{equation} \label{eq:context}
\begin{split}
	{context}_{attributes} = \{(budget (budget buyer), intent (sports), temperature (hot)), &\\
	                                            ((w_{budget}(0.7), w_{intent}(0.6), w_{temperature}(0.9))\} &
\end{split}
\end{equation}

$I$ is the data structure that describes the clothing item bought by the user. It is represented as a feature value weight vector (Equation xx.xx). It is directly borrowed from the base system introduced in section xx.xx. 

Through this case model, knowledge about what kind of items users buy in a certain context situation can be obtained. To provide recommendation, cases with context situation similar to the current user can be retrieved and the items contained in those cases can be used directly for recommendation. They can also be used as reference items to find other similar items to recommend. Thus we are going to show how the similarity between current context and retrieved cases and similarity between items are calculated.

\section{Similarity Assessment} \label{sec:sa}

To get the similarity between current context and retrieved cases, we borrow the Euclidean Overlap Metric (HEOM) [26, 34]:
\begin{equation} \label{eq:heom}
heom(x,y)=\frac{1}{\sqrt{\sum^n_{i=1}w_i}}\sqrt{\sum^n_{i=1}w_id_i(x_i,y_i)^2}
\end{equation}
where:

\textbf{TODO the equation}

Here $range_i$ is the difference between the maximum and minimum value of a numeric feature, and $overlap(x_i, y_i)=1$ if $x_i \neq y_i$ and 0 otherwise. This metric measures the distance between two vectors. Thus the further away two vectors, the more similar they are. We modified this metric so that it can be applied to our system.

By using the previously discussed case model and query structure, the feature value vectors of context factors describing the context situation in both structures can be fed into the similarity metric. First, for all the context factors submitted in the user? query (currentContext), we will calculate the similarity to the target context factors (targetContext) in the cases. In some cases, the targetContext may not contain context factors that are specified by the user (e.g., the user enables the context factor shopping intent and budget, but the targetContext only contains budget). In some other cases, the currentContext may not contain some context factors contained in the targetContext. So when computing similarities, we use user specified context factors as base and only consider the similarities between context factors specified by the user. The context factors contained in the targetContext but not in the currentContext will be ignored, because the user chooses to ignore those factors. If the targetContext does not contain the context factors specified in currentContext, the similarity will be added by $1* w_i$. ($w_i$ here corresponds to the feature factor weight).

The simplified similarity metric is displayed as follows:
\begin{lstlisting}[frame=single,caption=The simplified similarity metric]
define getSimilarity(query, case)
    targetContext = getCaseContext(case)
    currentContext = getQueryContext(query)

    for all context factors defined in targetConetxt
        if the targetContext contains the current context factor 
        in currentContext
            sim += factorSimilarity($\pi_f$(targetContext),
            	$\pi_f$(currentContext))
            weight += getWeight(case,$\pi_f$(targetContext))
        else
            sim += getWeight(case,$\pi_f$(targetContext))
            weight += getWeight(case,$\pi_f$(targetContext))
    endfor
    sim = Math.sqrt(sim/weight)
return sim
\end{lstlisting}




















