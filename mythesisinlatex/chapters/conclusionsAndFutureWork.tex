\chapter{Conclusions and Future Work}\label{chapter:conclusions}

In this thesis, a context-aware recommender system that integrates contextual information into existing Active Learning recommender system using case-based recommendation approach was developed and evaluated.

Following the methodology developed for developing context-aware recommender systems \cite{ref:5}, relevance of contextual factors was first acquired using a web tool developed to ask users to rate the influence of different context factors on their purchasing decisions by imagining themselves being in a selected contextual condition. After the relevance was obtained, a case-based recommendation approach was proposed to integrate contextual information into the recommender system by recommending clothes items bought by other users under similar contextual conditions to the current user. A case base was set up as the knowledge base for the recommendation using expert-driven approach. To solve the static suggestion ability problem of knowledge recommender system, collaborative filtering approach was used for long-term knowledge engineering. However, no privacy issue needs to be worried about because the correlation between users is performed at the session level and no user identification is required. To compute the similarity between retrieved case and user's submitted query, the Euclidean Overlap Metric (HEOM) was borrowed. Using the learned context factor relevance before, contextual explanation for each recommended item can be easily generated through finding out the most influential factors for the current type of the recommended clothes.

An Android mobile application Shopper using the developed recommender system was developed and evaluated and was shown to have a better evaluation regarding prediction accuracy, decision effort and general satisfaction compared to the baseline system that is not context-aware. Also users were shown to be well aware of the benefits of contextual information and expressed a strong demand for context-aware recommendation in the evaluation. 

TODO ... More about evaluation ...

Several things can be done for future improvement. From the application design perspective, users can be allowed to control and specify values for as many context factors as possible. In the system developed in this thesis, values of part of the context factors such as weather or temperature are automatically detected if the users enabled them in the context setting interface. However, during the evaluation, some users were complaining that the system was too intelligent by detecting the contextual information automatically and thus requirements for buying for future contextual scenarios can not be fulfilled. 

For the proposed recommendation approach, clothes items are recommended as individual items to the user. In the future, recommendation of clothes set can be realized through extending the current case structure by including more than one clothes item in each case. For example, a user who bought a coat in cold temperature may also buy a scarf that can match with the coat. Then this scarf can be included in the same case with the coat and be recommended as a set to another user.

The recommendation quality can also be improved through extending the case model by including an evaluation model. Right now, a case is stored in the case base if the user buys an item and the bought item is automatically evaluated as an ideal item for the corresponding context. In the future, more approaches can be adopted for the evaluation of the idealness of an item. The item viewed by the user, the item positively critiqued by the user, the stores most often visited by the user can also be used as hints for the evaluation. 

Regarding the commercialization of the application, real data set instead of the simulating one should be obtained from real offline stores, including the realtime stock information for each clothes item, the crowdedness of the store, etc. However, this can be difficult because retailers are usually quite sensitive to the exposition of such data. This then becomes a business model problem. Furthermore, currently for evaluation purpose, the recommendation calculation is done on the mobile device and the database is also stored locally. In the future, they should all be migrated to a remote server to reduce the burden of the client app and increase the system efficiency.

